\section{Introduction}

The Boolean satisfiability problem (SAT) consists of determining an assignment of variables such that a given Boolean expression evaluates to True, or proving no such expression exists; It is a very famous problem in logic and computer science since it was the first problem proven to be NP-Complete, thus finding an efficient general SAT solver is bound to advance the P vs NP problem, which is considered by many to be one of the most important open problems in computer science and is actually one of the unsolved\footnote{The Poincaré Conjecture is the only solved millennium problem} seven millennium problems stated by the Clay Mathematics Institute on May 24, 2000. 
\\This work attempts to study the behavior of genetic algorithms applied to a restricted version of the SAT problem called the 3-SAT problem, which is still NP-complete. It will attempt to do so by analyzing the behavior of genetic algorithms based on different operators and hyper-parameters tested on a series of 3-SAT problems, and then grading them against each other, the best ranking genetic algorithms found will then be compared with some of the most popular open source SAT solvers.

\subsection {Motivation}

A genetic algorithm is a meta-heuristic designed to mimic the behavior of natural selection, it does so by evolving a population of candidate solutions through the use of biologically inspired operators, it is a population based random search algorithm \parencite{Cochran2011}
TODO: ADD MORE MOTIVATION

\subsection {Objectives}


\subsection {Methodology}


\subsection {Document Structure}

